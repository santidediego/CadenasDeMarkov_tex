\documentclass{beamer}
\usepackage[spanish]{babel}
\usepackage[utf8]{inputenc}
\usetheme{AnnArbor}
\usecolortheme{crane}
\useoutertheme{shadow}
\useinnertheme{rectangles}
\usepackage{multicol}
\title[CMTC]{Cadenas de Markov en tiempo continuo}
\author[Santiago,Jesus,Fernando]{Santiago de Diego, Jesús Bueno, Fernando de la Cruz, Javier Ruiz}
\date{}
\begin{document}
\frame{\titlepage}

\AtBeginSection{
\begin{frame}
  \frametitle{Índice}
  \begin{multicols}{2}
  \tableofcontents[currentsection]
  \end{multicols}   
\end{frame}
}

\AtBeginSubsection{
\begin{frame}
  \frametitle{Índice}
  \begin{multicols}{2}
  \tableofcontents[currentsection,currentsubsection]
  \end{multicols}
\end{frame}
}

\section{Introducción}
\begin{frame}
    \frametitle{Introducción}
    Esto es una introducción
       \begin{block}{Mi bloque}
  		Esto es un bloque
  		\end{block}
\end{frame}

\section{Definición y propiedades}
\begin{frame}
    \frametitle{Definición y propiedades}
    Esto es una introducción
\end{frame}

\section{Probabilidades de transición}
\begin{frame}
    \frametitle{Probabilidades de transición}
    Esto es una introducción
\end{frame}

\section{Ecuación de Kolmogorov}
\begin{frame}
    \frametitle{Ecuación de Kolmogorov}
    Esto es una introducción
\end{frame}

\section{Clasificación de los estados}
\begin{frame}
    \frametitle{Clasificación de los estados}
    Esto es una introducción
\end{frame}

\section{Teoremas límite}
\begin{frame}
    \frametitle{Teoremas límite}
    Esto es una introducción
\end{frame}

\section{Ejemplos}
\subsection{Ejemplo 1}
\begin{frame}
    \frametitle{Ejemplos}
    Esto es una introducción
\end{frame}
\begin{frame}
    \frametitle{Ejemplo 1}
    Esto es una introducción
\end{frame}
\end{document}
